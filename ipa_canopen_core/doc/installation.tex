\chapter{Installation}
\label{chap:installation}

At the moment, the first thing you have to do (if not done already) is to manually install the CAN device driver, as described in Section \ref{chap:installation:devicedriver}.

For the installation of the IPA CANopen library, there are two options:
\begin{itemize}
\item For usage from within ROS (Section \ref{chap:installation:ipa_canopen_ros})
\item For using the C++ library without ROS (Section \ref{chap:installation:ipa_canopen_core})
\end{itemize}

\section{Prerequisites}

You will need the following free tools, which are available for all operating systems:
\begin{itemize}
\item {\em CMake} (to manage the build process). It is pre-installed on many *nix operating systems. Otherwise, you need to install it first, e.g. in Ubuntu:
\texttt{sudo apt-get install cmake}
\item {\em git} (to download the sources from github). In Ubuntu, it can be installed with: \texttt{sudo apt-get install git}
\item A C++ compiler with good support for the C++11 standard, e.g. {\em gcc} version 2.6 or higher (default in Ubuntu versions 11.04 or higher).
\end{itemize}

\section{CAN device driver}
\label{chap:installation:devicedriver}

Currently, the library has only been tested with the PCAN-USB CAN interface for USB from Peak System. It has been tested with version 7.6. The Linux user manual for the Peak interface is available at: \url{http://www.peak-system.com/fileadmin/media/linux/files/PCAN%20Driver%20for%20Linux_eng_7.1.pdf}. Briefly, to install the drivers under Linux, proceed as follows:
  \begin{itemize}
  \item Download and unpack the driver: \url{http://www.peak-system.com/fileadmin/media/linux/files/peak-linux-driver-7.6.tar.gz}.
  \item \texttt{cd peak-linux-driver-x.y}
  \item \texttt{make clean}
  \item Use the chardev driver: \texttt{make NET=NO}\\
    {\bf Note: The option \texttt{NET=no} is crucial!}
  \item \texttt{sudo make install}
  \item \texttt{/sbin/modprobe pcan}
  
  If the Kernel is newer than 3.5 the following driver should be downloaded:
  \url{http://www.peak-system.com/fileadmin/media/linux/files/peak-linux-driver-7.7.tar.gz}

  \item Test that the driver is working:
    \begin{itemize}
    \item \texttt{cat /proc/pcan} should look like this, especially \texttt{ndev} should be \texttt{NA}:
      {\scriptsize
\begin{verbatim}
*------------- PEAK-System CAN interfaces (www.peak-system.com) -------------
*-------------------------- Release_20120319_n (7.5.0) ----------------------
*---------------- [mod] [isa] [pci] [dng] [par] [usb] [pcc] -----------------
*--------------------- 1 interfaces @ major 248 found -----------------------
*n -type- ndev --base-- irq --btr- --read-- --write- --irqs-- -errors- status
32    usb -NA- ffffffff 255 0x001c 0000cc3f 0000edd1 00063ce1 00000005 0x0014
\end{verbatim}}
\item \texttt{./receivetest -f=/dev/pcan32} Turning the CAN device power on and off should trigger some CAN messages which should be shown on screen.
\item {\bf Note: If you do not receive messages using this test, you still have issues with the device driver which need to be solved before proceeding further.}
    \end{itemize}
  \end{itemize}

  \section{IPA CANopen ROS package}
\label{chap:installation:ipa_canopen_ros}

  \begin{itemize}
  \item \texttt{git clone \url{git://github.com/ipa320/ipa_canopen.git}}
  \item \texttt{rosmake ros\_canopen} 
  \item Test if the installation was successful:
    \begin{itemize}
    \item In one terminal: \texttt{roscore}
    \item In another terminal: \texttt{rosrun ipa\_canopen\_ros canopen\_ros}. This should give the following error message:
      {\scriptsize
\begin{verbatim}Missing parameters on parameter server; shutting down node.\end{verbatim}}
\end{itemize}
\item Optionally, you can also install the tutorial examples:
\begin{itemize}
\item \texttt{git clone \url{git://github.com/ipa320/ipa_canopen_tutorials.git}}
\item \texttt{rosmake ipa\_canopen\_tutorials}
\end{itemize}

\end{itemize}


 \section{ROS-indendent CANopen library}
\label{chap:installation:ipa_canopen_core}

If you just want to use the C++ library independently from ROS, follow the steps described below:

\subsection{Installation}

\begin{itemize}
\item Go to a directory in which you want to create the source directory.
\item \texttt{git clone \url{git://github.com/ipa320/ipa_canopen.git}}
\item \texttt{cd ipa\_canopen/ipa\_canopen\_core}
\item Create a build directory and enter it: \texttt{mkdir build \&\& cd build}
\item Prepare the make files: \texttt{cmake ..}
\item \texttt{make}
\begin{itemize}
\item Optionally, you can make the installation available system-wide: \\
\texttt{sudo make install}
\end{itemize}
\item Test if the build was successful:
\begin{itemize} 
\item \texttt{cd tools}
\item \texttt{./homing}
\item This should give the output:
{\scriptsize
\begin{verbatim}
      Arguments:
      (1) device file
      (2) CAN deviceID
      Example: ./homing /dev/pcan32 12
\end{verbatim}}
\end{itemize}
\end{itemize}

